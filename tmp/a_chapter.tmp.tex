\chapter{The first chapter}

\label{cha:a_chapter}

This is the first paragraph of the Softcover Markdown template produced with the \softcover\ command-line interface. It shows how to write a document in Markdown, a lightweight markup language, augmented with the \href{http://kramdown.gettalong.org/}{kramdown} converter and some custom extensions, including support for embedded \PolyTeX, a subset of the powerful \LaTeX\ typesetting system.\footnote{Pronunciations of ``LaTeX'' differ, but \emph{lay}-tech is the one I prefer.} For more information, see \href{http://manual.softcover.io/book}{\emph{The Softcover Book}}. To learn how to easily publish (and optionally sell) documents produced with Softcover, visit \href{http://softcover.io/}{Softcover.io}.

This is the \emph{second} paragraph, showing how to emphasize text.\footnote{This is a footnote. It is numbered automatically.} You can also make text \textbf{bold} or \emph{emphasize a second way}. Via embedded \PolyTeX, Softcover also supports colored text, such as \coloredtext{red}{red}, \coloredtext{CornflowerBlue}{cornflower blue}, and \coloredtexthtml{E8AB3A}{arbitrary HTML colors}.

\section{A section}

\label{sec:a_section}

This is a section. You can refer to it using the \LaTeX\ cross-reference syntax, like so: \hyperref[sec:a_section]{Section~\ref{sec:a_section}}.

\subsection{Source code}

This is a subsection.

You can typeset code samples and other verbatim text using four spaces of indentation:

\begin{verbatim}
def hello
  puts "hello, world"
end
\end{verbatim}

Softcover also comes with full support for syntax-highlighted source code using kramdown's default syntax, which combines the language name with indentation:

\begin{framed_shaded}
\begin{Verbatim}[fontsize=\relsize{-2.5},fontseries=b,commandchars=\\\{\}]
\PY{k}{def} \PY{n+nf}{hello}
  \PY{n+nb}{puts} \PY{l+s+s2}{\PYZdq{}}\PY{l+s+s2}{hello, world}\PY{l+s+s2}{\PYZdq{}}
\PY{k}{end}
\end{Verbatim}
\end{framed_shaded}

Softcover's Markdown mode also extends kramdown to support ``code fencing'' from GitHub-flavored Markdown:

\begin{framed_shaded}
\begin{Verbatim}[fontsize=\relsize{-2.5},fontseries=b,commandchars=\\\{\}]
\PY{k}{def} \PY{n+nf}{hello}
  \PY{n+nb}{puts} \PY{l+s+s2}{\PYZdq{}}\PY{l+s+s2}{hello, world!}\PY{l+s+s2}{\PYZdq{}}
\PY{k}{end}
\end{Verbatim}
\end{framed_shaded}

The last of these can be combined with \PolyTeX's \kode{codelisting} environment to make code listings with linked cross-references (\hyperref[code:hello]{Listing~\ref{code:hello}}).

\begin{codelisting}
\codecaption{Hello, world.}
\label{code:hello}
\begin{Verbatim}[fontsize=\relsize{-2.5},fontseries=b,commandchars=\\\{\}]
\PY{k}{def} \PY{n+nf}{hello}
  \PY{n+nb}{puts} \PY{l+s+s2}{\PYZdq{}}\PY{l+s+s2}{hello, world!}\PY{l+s+s2}{\PYZdq{}}
\PY{k}{end}
\end{Verbatim}
\end{codelisting}

\subsection{Mathematics}

Softcover's Markdown mode supports mathematical typesetting using \LaTeX\ syntax, including inline math, such as \( \phi^2 - \phi - 1 = 0, \) and centered math, such as
\[ \phi = \frac{1+\sqrt{5}}{2}. \]
It also supports centered equations with linked cross-reference via embedded \PolyTeX\ (\hyperref[eq:phi]{Eq.~\eqref{eq:phi}}).

\begin{equation}
\label{eq:phi}
\phi = \frac{1+\sqrt{5}}{2}
\end{equation}

Softcover also supports an alternate math syntax, such as \(\phi^2 - \phi - 1 = 0\), and centered math, such as

\[\phi = \frac{1+\sqrt{5}}{2}.\]

The \LaTeX\ syntax is strongly preferred, but the alternate syntax is included for maximum compatibility with other systems.

\section{Images and tables}

This is the second section.

Softcover supports the inclusion of images, like this:

\image{images/01_michael_hartl_headshot.jpg}

Using \LaTeX\ labels, you can also include a caption (as in \hyperref[fig:captioned_image]{Figure~\ref{fig:captioned_image}}) or just a figure number (as in \hyperref[fig:figure_number]{Figure~\ref{fig:figure_number}}).

\begin{figure}[h]
\begin{center}
\image{images/01_michael_hartl_headshot.jpg}
\end{center}
\caption{Some dude.\label{fig:captioned_image}}

\end{figure}

\begin{figure}[h]
\begin{center}
\image{images/01_michael_hartl_headshot.jpg}
\end{center}
\caption{\label{fig:figure_number}}

\end{figure}

\subsection{Tables}

Softcover supports raw tables via a simple table syntax:

\begin{longtable}{|l|l|l|l|}
\hline
\textbf{HTTP request} & \textbf{URL} & \textbf{Action} & \textbf{Purpose}\\
\kode{GET} & /users & \kode{index} & page to list all users\\
\kode{GET} & /users/1 & \kode{show} & page to show user with id \kode{1}\\
\kode{GET} & /users/new & \kode{new} & page to make a new user\\
\kode{POST} & /users & \kode{create} & create a new user\\
\kode{GET} & /users/1/edit & \kode{edit} & page to edit user with id \kode{1}\\
\kode{PATCH} & /users/1 & \kode{update} & update user with id \kode{1}\\
\kode{DELETE} & /users/1 & \kode{destroy} & delete user with id \kode{1}\\
\hline
\end{longtable}

See \href{http://manual.softcover.io/book/softcover_markdown#sec-embedded_tabular_and_tables}{\emph{The Softcover Book}} to learn how to make more complicated tables.

\section{Command-line interface}

Softcover comes with a command-line interface called \kode{softcover}. To get more information, just run \kode{softcover help}:

\begin{framed_shaded}
\begin{Verbatim}[fontsize=\relsize{-2.5},fontseries=b,commandchars=\\\{\}]
\PY{g+gp}{\PYZdl{}} softcover \PY{n+nb}{help}
\PY{g+go}{Commands:}
\PY{g+go}{  softcover build, build:all           \PYZsh{} Build all formats}
\PY{g+go}{  softcover build:epub                 \PYZsh{} Build EPUB}
\PY{g+go}{  softcover build:html                 \PYZsh{} Build HTML}
\PY{g+go}{  softcover build:mobi                 \PYZsh{} Build MOBI}
\PY{g+go}{  softcover build:pdf                  \PYZsh{} Build PDF}
\PY{g+go}{  softcover build:preview              \PYZsh{} Build book preview in all formats}
\PY{g+go}{  .}
\PY{g+go}{  .}
\PY{g+go}{  .}
\end{Verbatim}
\end{framed_shaded}

\noindent You can run \kode{softcover help \textless{}command\textgreater{}} to get additional help on a given command:

\begin{framed_shaded}
\begin{Verbatim}[fontsize=\relsize{-2.5},fontseries=b,commandchars=\\\{\}]
\PY{g+gp}{\PYZdl{}} softcover \PY{n+nb}{help }build
\PY{g+go}{Usage:}
\PY{g+go}{  softcover build, build:all}

\PY{g+go}{Options:}
\PY{g+go}{  \PYZhy{}q, [\PYZhy{}\PYZhy{}quiet]   \PYZsh{} Quiet output}
\PY{g+go}{  \PYZhy{}s, [\PYZhy{}\PYZhy{}silent]  \PYZsh{} Silent output}

\PY{g+go}{Build all formats}
\end{Verbatim}
\end{framed_shaded}

\section{Miscellanea}

This is the end of the template---apart from two mostly empty chapters. In fact, let’s include the last chapter in its entirety, just to see how mostly empty it is:
\begin{framed_shaded}
\begin{Verbatim}[fontsize=\relsize{-2.5},fontseries=b,commandchars=\\\{\}]
\PYZsh{} Yet *another* chapter

*This chapter left intentionally blank*
\end{Verbatim}
\end{framed_shaded}

Visit \href{http://manual.softcover.io}{\emph{The Softcover Book}} to learn more about what Softcover can do.