\chapter{Chef}

\section{Chef components}

\section{Knife}

\section{OHAI}

\section{Atributes}

\section{Databags}

\section{Environments}

\section{LIghtweight resources and providers}

\section{Cookbooks}

\section{Recipes}

\section{Install CHEF}

The first step is to check the server host name.

\begin{codelisting}
\label{code:hostname}
\codecaption{}
\begin{Verbatim}[fontsize=\relsize{-2.5},fontseries=b,commandchars=\\\{\}]
root@chef01:\PYZti{}\PYZsh{} hostname \PYZhy{}f
\end{Verbatim}
\end{codelisting}

Now we need to edit the host name configuration file in order to add the FQDN of the server.
\begin{codelisting}
\label{code:}
\codecaption{}
\begin{Verbatim}[fontsize=\relsize{-2.5},fontseries=b,commandchars=\\\{\}]
root@chef01:\PYZti{}\PYZsh{} sudo nano /etc/hosts
\end{Verbatim}
\end{codelisting}

Your configuration file should be something similar to this one.

\image{images/figures/02_host_name.png}

Update the packages list and update the server.

\begin{codelisting}
\label{code:}
\codecaption{}
\begin{Verbatim}[fontsize=\relsize{-2.5},fontseries=b,commandchars=\\\{\}]
root@chef01:\PYZti{}\PYZsh{} sudo aptitude update
root@chef01:\PYZti{}\PYZsh{} sudo aptitude upgrade
\end{Verbatim}
\end{codelisting}

Download the latest version of the chef server to the root folder of the current user.

\begin{codelisting}
\label{code:}
\codecaption{}
\begin{Verbatim}[fontsize=\relsize{-2.5},fontseries=b,commandchars=\\\{\}]
root@chef01:\PYZti{}\PYZsh{} wget https://web\PYZhy{}dl.packagecloud.io/chef/stable/packages/ubuntu/trusty/chef\PYZhy{}server\PYZhy{}core\PYZus{}12.2.0\PYZhy{}1\PYZus{}amd64.deb
\end{Verbatim}
\end{codelisting}

Install the Chef Server.

\begin{codelisting}
\label{code:}
\codecaption{}
\begin{Verbatim}[fontsize=\relsize{-2.5},fontseries=b,commandchars=\\\{\}]
root@chef01:\PYZti{}\PYZsh{} sudo dpkg \PYZhy{}i chef\PYZhy{}server\PYZhy{}core\PYZus{}*.deb
root@chef01:\PYZti{}\PYZsh{} sudo chef\PYZhy{}server\PYZhy{}ctl reconfigure
\end{Verbatim}
\end{codelisting}

\image{images/figures/06_install_chef_api.PNG}

This should be the result\ldots{}.

Install additional modules.

\image{images/figures/07_install_chef_web_ui.PNG}

(Premium freature up to 25 nodes..)
\begin{codelisting}
\label{code:}
\codecaption{}
\begin{Verbatim}[fontsize=\relsize{-2.5},fontseries=b,commandchars=\\\{\}]
root@chef01:\PYZti{}\PYZsh{} chef\PYZhy{}server\PYZhy{}ctl install opscode\PYZhy{}manage\PY{p}{;}
root@chef01:\PYZti{}\PYZsh{} chef\PYZhy{}server\PYZhy{}ctl install opscode\PYZhy{}push\PYZhy{}jobs\PYZhy{}server\PY{p}{;}
root@chef01:\PYZti{}\PYZsh{} chef\PYZhy{}server\PYZhy{}ctl install opscode\PYZhy{}reporting\PY{p}{;}
root@chef01:\PYZti{}\PYZsh{} opscode\PYZhy{}manage\PYZhy{}ctl reconfigure
root@chef01:\PYZti{}\PYZsh{} opscode\PYZhy{}push\PYZhy{}jobs\PYZhy{}server\PYZhy{}ctl reconfigure\PY{p}{;}
root@chef01:\PYZti{}\PYZsh{} opscode\PYZhy{}reporting\PYZhy{}ctl reconfigure\PY{p}{;} 
root@chef01:\PYZti{}\PYZsh{} chef\PYZhy{}server\PYZhy{}ctl reconfigure\PY{p}{;}
\end{Verbatim}
\end{codelisting}

We need to create the first user (Admin user)

\begin{codelisting}
\label{code:}
\codecaption{}
\begin{Verbatim}[fontsize=\relsize{-2.5},fontseries=b,commandchars=\\\{\}]
root@chef01:\PYZti{}\PYZsh{} chef\PYZhy{}server\PYZhy{}ctl user\PYZhy{}create admin the administrator the\PYZus{}good@chefbyexample.com 4dm1n1str4t0r \PYZhy{}f admin.pem
\end{Verbatim}
\end{codelisting}

WE need to create the first organization and add the first created user to it.

\begin{codelisting}
\label{code:}
\codecaption{}
\begin{Verbatim}[fontsize=\relsize{-2.5},fontseries=b,commandchars=\\\{\}]
root@chef01:\PYZti{}\PYZsh{} sudo chef\PYZhy{}server\PYZhy{}ctl org\PYZhy{}create chefbyexample \PY{l+s+s2}{\PYZdq{}ChefByExample.com\PYZdq{}} \PYZhy{}\PYZhy{}association\PYZus{}user admin \PYZhy{}f chefbyexample\PYZhy{}validator.pem
\end{Verbatim}
\end{codelisting}

\image{images/figures/12_install_chef_finished.PNG}

Now the Chef SErver is fully operative, now we need to add the WOrkstation.

\section{Install the Workstation}

\begin{codelisting}
\label{code:}
\codecaption{}
\begin{Verbatim}[fontsize=\relsize{-2.5},fontseries=b,commandchars=\\\{\}]
sudo wget https://opscode\PYZhy{}omnibus\PYZhy{}packages.s3.amazonaws.com/ubuntu/12.04/x86\PYZus{}64/chefdk\PYZus{}0.7.0\PYZhy{}1\PYZus{}amd64.deb

sudo dpkg \PYZhy{}i chefdk\PYZus{}*.deb

sudo chef generate repo chef\PYZhy{}repo

mkdir \PYZti{}/chef\PYZhy{}repo/.chef
scp root@chef01.chefbyexample.com:/root/admin.pem \PYZti{}/chef\PYZhy{}repo/.chef
scp root@chef01.chefbyexample.com:/root/chefbyexample\PYZhy{}validator.pem \PYZti{}/chef\PYZhy{}repo/.chef

nano \PYZti{}/chef\PYZhy{}repo/.chef/knife.rb


\PY{n+nv}{current\PYZus{}dir} \PY{o}{=} File.dirname\PY{o}{(}\PYZus{}\PYZus{}FILE\PYZus{}\PYZus{}\PY{o}{)}
log\PYZus{}level                :info
log\PYZus{}location             STDOUT
node\PYZus{}name                \PY{l+s+s2}{\PYZdq{}admin\PYZdq{}}
client\PYZus{}key               \PY{l+s+s2}{\PYZdq{}\PYZsh{}\PYZob{}current\PYZus{}dir\PYZcb{}/admin.pem\PYZdq{}}
validation\PYZus{}client\PYZus{}name   \PY{l+s+s2}{\PYZdq{}chefbyexample\PYZhy{}validator\PYZdq{}}
validation\PYZus{}key           \PY{l+s+s2}{\PYZdq{}\PYZsh{}\PYZob{}current\PYZus{}dir\PYZcb{}/chefbyexample\PYZhy{}validator.pem\PYZdq{}}
chef\PYZus{}server\PYZus{}url          \PY{l+s+s2}{\PYZdq{}https://chef01.chefbyexample.com/organizations/chefbyexample\PYZdq{}}
syntax\PYZus{}check\PYZus{}cache\PYZus{}path  \PY{l+s+s2}{\PYZdq{}\PYZsh{}\PYZob{}ENV[\PYZsq{}HOME\PYZsq{}]\PYZcb{}/.chef/syntaxcache\PYZdq{}}
cookbook\PYZus{}path            \PY{o}{[}\PY{l+s+s2}{\PYZdq{}\PYZsh{}\PYZob{}current\PYZus{}dir\PYZcb{}/../cookbooks\PYZdq{}}\PY{o}{]}


knife ssl fetch

Bootstraping the first node

knife bootstrap node01.chefbyexample.com \PYZhy{}N node01
\end{Verbatim}
\end{codelisting}

\section{Installing the Open Source Web Interface}

\subsection{Requirements installation}

Installation steps, just run:
\begin{codelisting}
\label{code:}
\codecaption{}
\begin{Verbatim}[fontsize=\relsize{-2.5},fontseries=b,commandchars=\\\{\}]
aptitude install git rubygems1.9.1 ruby1.9.1\PYZhy{}dev build\PYZhy{}essential
mkdir \PYZhy{}p /var/www/
\PY{n+nb}{cd} /var/www/
git clone https://github.com/carlosdcg/chef\PYZhy{}server\PYZhy{}webui
\PY{n+nb}{cd }chef\PYZhy{}server\PYZhy{}webui
gem install bundler
bundle install
\end{Verbatim}
\end{codelisting}

\subsection{COnfigure the default paramenters}

Configure the web app in /var/www/chef-server-webui/config/application.rb

\begin{codelisting}
\label{code:}
\codecaption{}
\begin{Verbatim}[fontsize=\relsize{-2.5},fontseries=b,commandchars=\\\{\}]
\PY{n}{config}\PY{o}{.}\PY{n}{chef\PYZus{}server\PYZus{}url} \PY{o}{=} \PY{l+s+s2}{\PYZdq{}}\PY{l+s+s2}{http://127.0.0.1}\PY{l+s+s2}{\PYZdq{}}
\PY{n}{config}\PY{o}{.}\PY{n}{rest\PYZus{}client\PYZus{}name} \PY{o}{=} \PY{l+s+s2}{\PYZdq{}}\PY{l+s+s2}{pivotal}\PY{l+s+s2}{\PYZdq{}}
\PY{n}{config}\PY{o}{.}\PY{n}{rest\PYZus{}client\PYZus{}key} \PY{o}{=} \PY{l+s+s2}{\PYZdq{}}\PY{l+s+s2}{/etc/opscode/pivotal.pem}\PY{l+s+s2}{\PYZdq{}}
\PY{n}{config}\PY{o}{.}\PY{n}{admin\PYZus{}user\PYZus{}name} \PY{o}{=}  \PY{l+s+s2}{\PYZdq{}}\PY{l+s+s2}{admin}\PY{l+s+s2}{\PYZdq{}}
\PY{n}{config}\PY{o}{.}\PY{n}{admin\PYZus{}default\PYZus{}password} \PY{o}{=} \PY{l+s+s2}{\PYZdq{}}\PY{l+s+s2}{4dm1n1str4t0r}\PY{l+s+s2}{\PYZdq{}}
\PY{n}{config}\PY{o}{.}\PY{n}{rest\PYZus{}client\PYZus{}custom\PYZus{}http\PYZus{}headers} \PY{o}{=} \PY{p}{\PYZob{}}\PY{p}{\PYZcb{}}
\PY{c+c1}{\PYZsh{}This app only supports one organization, like the Open Source Chef Server 11}
\PY{n}{config}\PY{o}{.}\PY{n}{default\PYZus{}organization} \PY{o}{=} \PY{l+s+s2}{\PYZdq{}}\PY{l+s+s2}{organizations/chefbyexample/}\PY{l+s+s2}{\PYZdq{}}
\end{Verbatim}
\end{codelisting}

\subsection{Use the interface on demand or install it as a service}

Once the Web UI is installed, from /var/www/chef-server-webui run:

To test in the default port 9292:
\begin{codelisting}
\label{code:}
\codecaption{}
\begin{Verbatim}[fontsize=\relsize{-2.5},fontseries=b,commandchars=\\\{\}]
rackup config.ru
\end{Verbatim}
\end{codelisting}

To run as a daemon in another port to test:
\begin{codelisting}
\label{code:}
\codecaption{}
\begin{Verbatim}[fontsize=\relsize{-2.5},fontseries=b,commandchars=\\\{\}]
rackup config.ru \PYZhy{}D \PYZhy{}p 1234
\end{Verbatim}
\end{codelisting}

Once you have tested it, to create the init scripts and install the run levels---
\begin{codelisting}
\label{code:}
\codecaption{}
\begin{Verbatim}[fontsize=\relsize{-2.5},fontseries=b,commandchars=\\\{\}]
\PY{c}{\PYZsh{}TO remove the script from the default run\PYZhy{}levels}
\PY{c}{\PYZsh{}sudo update\PYZhy{}rc.d \PYZhy{}f chef\PYZhy{}server\PYZhy{}webui remove}
sudo chmod \PY{l+m}{755} /var/www/chef\PYZhy{}server\PYZhy{}webui/init/chef\PYZhy{}server\PYZhy{}webui.sh
ln \PYZhy{}s /var/www/chef\PYZhy{}server\PYZhy{}webui/init/chef\PYZhy{}server\PYZhy{}webui.sh /etc/init.d/chef\PYZhy{}server\PYZhy{}webui
sudo chmod \PY{l+m}{755} /etc/init.d/chef\PYZhy{}server\PYZhy{}webui
sudo chown root:root /etc/init.d/chef\PYZhy{}server\PYZhy{}webui
sudo update\PYZhy{}rc.d chef\PYZhy{}server\PYZhy{}webui defaults
\end{Verbatim}
\end{codelisting}