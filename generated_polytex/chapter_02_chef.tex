\chapter{Chef}

\section{Chef components}

\section{Knife}

\section{OHAI}

\section{Atributes}

\section{Databags}

\section{Environments}

\section{LIghtweight resources and providers}

\section{Cookbooks}

\section{Recipes}

\section{Install CHEF}

\image{images/figures/01_host_name.png}

\begin{codelisting}
\label{code:hostname}
\codecaption{}
%= lang:bash
\begin{code}
root@chef01:~# hostname -f
\end{code}
\end{codelisting}

\image{images/figures/02_host_name.png}

\begin{codelisting}
\label{code:}
\codecaption{}
%= lang:bash
\begin{code}
root@chef01:~# sudo nano /etc/hosts
\end{code}
\end{codelisting}

\image{images/figures/03_install_chef.PNG}

\begin{codelisting}
\label{code:}
\codecaption{}
%= lang:bash
\begin{code}
root@chef01:~# wget https://web-dl.packagecloud.io/chef/stable/packages/ubuntu/trusty/chef-server-core_12.2.0-1_amd64.deb
\end{code}
\end{codelisting}

\image{images/figures/04_install_chef.PNG}

\begin{codelisting}
\label{code:}
\codecaption{}
%= lang:bash
\begin{code}
root@chef01:~# sudo dpkg -i chef-server-core_*.deb
\end{code}
\end{codelisting}

\image{images/figures/05_install_chef_config.PNG}

\begin{codelisting}
\label{code:}
\codecaption{}
%= lang:bash
\begin{code}
root@chef01:~# sudo chef-server-ctl reconfigure
\end{code}
\end{codelisting}

\image{images/figures/06_install_chef_api.PNG}

Web interface

\image{images/figures/07_install_chef_web_ui.PNG}

(Premium freature up to 25 nodes..)
\begin{codelisting}
\label{code:}
\codecaption{}
%= lang:bash
\begin{code}
root@chef01:~# chef-server-ctl install opscode-manage; chef-server-ctl reconfigure; opscode-manage-ctl reconfigure
\end{code}
\end{codelisting}

\image{images/figures/08_install_chef_push_jobs.PNG}

\begin{codelisting}
\label{code:}
\codecaption{}
%= lang:bash
\begin{code}
root@chef01:~# chef-server-ctl install opscode-push-jobs-server; chef-server-ctl reconfigure; opscode-push-jobs-server-ctl reconfigure;
\end{code}
\end{codelisting}

\image{images/figures/09_install_chef_reporting.PNG}

(Premium freature up to 25 nodes..)
\begin{codelisting}
\label{code:}
\codecaption{}
%= lang:bash
\begin{code}
root@chef01:~# chef-server-ctl install opscode-reporting; chef-server-ctl reconfigure; opscode-reporting-ctl reconfigure; 
\end{code}
\end{codelisting}

\image{images/figures/10_install_chef_add_user.PNG}

\begin{codelisting}
\label{code:}
\codecaption{}
%= lang:bash
\begin{code}
root@chef01:~# chef-server-ctl user-create admin the administrator the_good@chefbyexample.com 4dm1n1str4t0r -f admin.pem
\end{code}
\end{codelisting}

\image{images/figures/11_install_chef_add_org.PNG}

\begin{codelisting}
\label{code:}
\codecaption{}
%= lang:bash
\begin{code}
root@chef01:~# sudo chef-server-ctl org-create chefbyexample "ChefByExample.com" --association_user admin -f chefbyexample-validator.pem
\end{code}
\end{codelisting}

\image{images/figures/12_install_chef_finished.PNG}

Chef Server installed..

\section{Install one Workstation}

\begin{codelisting}
\label{code:}
\codecaption{}
%= lang:bash
\begin{code}

sudo wget https://opscode-omnibus-packages.s3.amazonaws.com/ubuntu/12.04/x86_64/chefdk_0.7.0-1_amd64.deb

sudo dpkg -i chefdk_*.deb

sudo chef generate repo chef-repo

mkdir ~/chef-repo/.chef
scp root@chef01.chefbyexample.com:/root/admin.pem ~/chef-repo/.chef
scp root@chef01.chefbyexample.com:/root/chefbyexample-validator.pem ~/chef-repo/.chef

nano ~/chef-repo/.chef/knife.rb


current_dir = File.dirname(__FILE__)
log_level                :info
log_location             STDOUT
node_name                "admin"
client_key               "#{current_dir}/admin.pem"
validation_client_name   "chefbyexample-validator"
validation_key           "#{current_dir}/chefbyexample-validator.pem"
chef_server_url          "https://chef01.chefbyexample.com/organizations/chefbyexample"
syntax_check_cache_path  "#{ENV['HOME']}/.chef/syntaxcache"
cookbook_path            ["#{current_dir}/../cookbooks"]


knife ssl fetch

Bootstraping the first node

knife bootstrap node01.chefbyexample.com -N node01


\end{code}
\end{codelisting}

\section{Install}

Installation steps, just run:
\begin{codelisting}
\label{code:}
\codecaption{}
%= lang:bash
\begin{code}
aptitude install git rubygems1.9.1 ruby1.9.1-dev build-essential
mkdir -p /var/www/
cd /var/www/
git clone https://github.com/carlosdcg/chef-server-webui
cd chef-server-webui
gem install bundler
bundle install
\end{code}
\end{codelisting}

Configure the web app in /var/www/chef-server-webui/config/application.rb

\begin{codelisting}
\label{code:}
\codecaption{}
%= lang:ruby
\begin{code}
config.chef_server_url = "http://127.0.0.1"
config.rest_client_name = "pivotal"
config.rest_client_key = "/etc/opscode/pivotal.pem"
config.admin_user_name =  "admin"
config.admin_default_password = "4dm1n1str4t0r"
config.rest_client_custom_http_headers = {}
#This app only supports one organization, like the Open Source Chef Server 11
config.default_organization = "organizations/chefbyexample/"
\end{code}
\end{codelisting}

\section{Use}

Once the Web UI is installed, from /var/www/chef-server-webui run:

To test in the default port 9292:
\begin{codelisting}
\label{code:}
\codecaption{}
%= lang:bash
\begin{code}
rackup config.ru
\end{code}
\end{codelisting}

To run as a daemon in another port:
\begin{codelisting}
\label{code:}
\codecaption{}
%= lang:bash
\begin{code}
rackup config.ru -D -p 1234
\end{code}
\end{codelisting}