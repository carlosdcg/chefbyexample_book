\chapter{Chef}

\section{Chef components}

\section{Knife}

\section{OHAI}

\section{Atributes}

\section{Databags}

\section{Environments}

\section{LIghtweight resources and providers}

\section{Cookbooks}

\section{Recipes}

\section{Install CHEF}

The first step is to check the server host name.

\begin{codelisting}
\label{code:hostname}
\codecaption{}
%= lang:bash
\begin{code}
root@chef01:~# hostname -f
\end{code}
\end{codelisting}

Now we need to edit the host name configuration file in order to add the FQDN of the server.
\begin{codelisting}
\label{code:}
\codecaption{}
%= lang:bash
\begin{code}
root@chef01:~# sudo nano /etc/hosts
\end{code}
\end{codelisting}

Your configuration file should be something similar to this one.

\image{images/figures/02_host_name.png}

Update the packages list and update the server.

\begin{codelisting}
\label{code:}
\codecaption{}
%= lang:bash
\begin{code}
root@chef01:~# sudo aptitude update
root@chef01:~# sudo aptitude upgrade
\end{code}
\end{codelisting}

Download the latest version of the chef server to the root folder of the current user.

\begin{codelisting}
\label{code:}
\codecaption{}
%= lang:bash
\begin{code}
root@chef01:~# wget https://web-dl.packagecloud.io/chef/stable/packages/ubuntu/trusty/chef-server-core_12.2.0-1_amd64.deb
\end{code}
\end{codelisting}

Install the Chef Server.

\begin{codelisting}
\label{code:}
\codecaption{}
%= lang:bash
\begin{code}
root@chef01:~# sudo dpkg -i chef-server-core_*.deb
root@chef01:~# sudo chef-server-ctl reconfigure
\end{code}
\end{codelisting}

\image{images/figures/06_install_chef_api.PNG}

This should be the result\ldots{}.

Install additional modules.

\image{images/figures/07_install_chef_web_ui.PNG}

(Premium freature up to 25 nodes..)
\begin{codelisting}
\label{code:}
\codecaption{}
%= lang:bash
\begin{code}
root@chef01:~# chef-server-ctl install opscode-manage;
root@chef01:~# chef-server-ctl install opscode-push-jobs-server;
root@chef01:~# chef-server-ctl install opscode-reporting;
root@chef01:~# opscode-manage-ctl reconfigure
root@chef01:~# opscode-push-jobs-server-ctl reconfigure;
root@chef01:~# opscode-reporting-ctl reconfigure; 
root@chef01:~# chef-server-ctl reconfigure;

\end{code}
\end{codelisting}

We need to create the first user (Admin user)

\begin{codelisting}
\label{code:}
\codecaption{}
%= lang:bash
\begin{code}
root@chef01:~# chef-server-ctl user-create admin the administrator the_good@chefbyexample.com 4dm1n1str4t0r -f admin.pem
\end{code}
\end{codelisting}

WE need to create the first organization and add the first created user to it.

\begin{codelisting}
\label{code:}
\codecaption{}
%= lang:bash
\begin{code}
root@chef01:~# sudo chef-server-ctl org-create chefbyexample "ChefByExample.com" --association_user admin -f chefbyexample-validator.pem
\end{code}
\end{codelisting}

\image{images/figures/12_install_chef_finished.PNG}

Now the Chef SErver is fully operative, now we need to add the WOrkstation.

\section{Install the Workstation}

\begin{codelisting}
\label{code:}
\codecaption{}
%= lang:bash
\begin{code}

sudo wget https://opscode-omnibus-packages.s3.amazonaws.com/ubuntu/12.04/x86_64/chefdk_0.7.0-1_amd64.deb

sudo dpkg -i chefdk_*.deb

sudo chef generate repo chef-repo

mkdir ~/chef-repo/.chef
scp root@chef01.chefbyexample.com:/root/admin.pem ~/chef-repo/.chef
scp root@chef01.chefbyexample.com:/root/chefbyexample-validator.pem ~/chef-repo/.chef

nano ~/chef-repo/.chef/knife.rb


current_dir = File.dirname(__FILE__)
log_level                :info
log_location             STDOUT
node_name                "admin"
client_key               "#{current_dir}/admin.pem"
validation_client_name   "chefbyexample-validator"
validation_key           "#{current_dir}/chefbyexample-validator.pem"
chef_server_url          "https://chef01.chefbyexample.com/organizations/chefbyexample"
syntax_check_cache_path  "#{ENV['HOME']}/.chef/syntaxcache"
cookbook_path            ["#{current_dir}/../cookbooks"]


knife ssl fetch

Bootstraping the first node

knife bootstrap node01.chefbyexample.com -N node01


\end{code}
\end{codelisting}

\section{Installing the Open Source Web Interface}

\subsection{Requirements installation}

Installation steps, just run:
\begin{codelisting}
\label{code:}
\codecaption{}
%= lang:bash
\begin{code}
aptitude install git rubygems1.9.1 ruby1.9.1-dev build-essential
mkdir -p /var/www/
cd /var/www/
git clone https://github.com/carlosdcg/chef-server-webui
cd chef-server-webui
gem install bundler
bundle install
\end{code}
\end{codelisting}

\subsection{COnfigure the default paramenters}

Configure the web app in /var/www/chef-server-webui/config/application.rb

\begin{codelisting}
\label{code:}
\codecaption{}
%= lang:ruby
\begin{code}
config.chef_server_url = "http://127.0.0.1"
config.rest_client_name = "pivotal"
config.rest_client_key = "/etc/opscode/pivotal.pem"
config.admin_user_name =  "admin"
config.admin_default_password = "4dm1n1str4t0r"
config.rest_client_custom_http_headers = {}
#This app only supports one organization, like the Open Source Chef Server 11
config.default_organization = "organizations/chefbyexample/"
\end{code}
\end{codelisting}

\subsection{Use the interface on demand or install it as a service}

Once the Web UI is installed, from /var/www/chef-server-webui run:

To test in the default port 9292:
\begin{codelisting}
\label{code:}
\codecaption{}
%= lang:bash
\begin{code}
rackup config.ru
\end{code}
\end{codelisting}

To run as a daemon in another port to test:
\begin{codelisting}
\label{code:}
\codecaption{}
%= lang:bash
\begin{code}
rackup config.ru -D -p 1234
\end{code}
\end{codelisting}

Once you have tested it, to create the init scripts and install the run levels---
\begin{codelisting}
\label{code:}
\codecaption{}
%= lang:bash
\begin{code}
#TO remove the script from the default run-levels
#sudo update-rc.d -f chef-server-webui remove
sudo chmod 755 /var/www/chef-server-webui/init/chef-server-webui.sh
ln -s /var/www/chef-server-webui/init/chef-server-webui.sh /etc/init.d/chef-server-webui
sudo chmod 755 /etc/init.d/chef-server-webui
sudo chown root:root /etc/init.d/chef-server-webui
sudo update-rc.d chef-server-webui defaults
\end{code}
\end{codelisting}