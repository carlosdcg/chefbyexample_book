\chapter{The first chapter}

\label{cha:a_chapter}

\begin{itemize}
\item General configuration structure for the chef-repo:


\begin{itemize}
\item chef-repo/environments/banana.rb
\item chef-repo/environments/potato.rb
\item chef-repo/environments/kiwi.rb
\item chef-repo/data\_bags/banana.rb
\item chef-repo/data\_bags/potato.rb
\item chef-repo/data\_bags/kiwi.rb: 
\item chef-repo/roles/base.rb
\item chef-repo/roles/web.rb
\item chef-repo/roles/db.rb
\end{itemize}
\item Banana cookbook structure:


\begin{itemize}
\item chef-repo/cookbooks/banana/templates/default/*.erb
\item chef-repo/cookbooks/banana/attributes/default.rb
\end{itemize}
\item Potato cookbook structure:


\begin{itemize}
\item chef-repo/cookbooks/potato/templates/default/*.erb
\item chef-repo/cookbooks/potato/attributes/default.rb
\end{itemize}
\end{itemize}

-Kiwi cookbook structure:
    * chef-repo/cookbooks/kiwi/templates/default/*.erb
    * chef-repo/cookbooks/kiwi/attributes/default.rb

A node belongs to an environment in which case, will override the default configuration per the corresponding one.

Override app attributes for kiwi (Non sensitive info)
Override app attributes for kiwi (Sensitive info)
Define the recipes for the xxx role

Default configuration templates for kiwi
Default configuration values according the templates for kiwi

This is the first paragraph of the Softcover Markdown template produced with the \softcover\ command-line interface. It shows how to write a document in Markdown, a lightweight markup language, augmented with the \href{http://kramdown.gettalong.org/}{kramdown} converter and some custom extensions, including support for embedded \PolyTeX, a subset of the powerful \LaTeX\ typesetting system.\footnote{Pronunciations of ``LaTeX'' differ, but \emph{lay}-tech is the one I prefer.} For more information, see \href{http://manual.softcover.io/book}{\emph{The Softcover Book}}. To learn how to easily publish (and optionally sell) documents produced with Softcover, visit \href{http://softcover.io/}{Softcover.io}.

This is the \emph{second} paragraph, showing how to emphasize text.\footnote{This is a footnote. It is numbered automatically.} You can also make text \textbf{bold} or \emph{emphasize a second way}. Via embedded \PolyTeX, Softcover also supports colored text, such as \coloredtext{red}{red}, \coloredtext{CornflowerBlue}{cornflower blue}, and \coloredtexthtml{E8AB3A}{arbitrary HTML colors}.

\section{A section}

\label{sec:a_section}

This is a section. You can refer to it using the \LaTeX\ cross-reference syntax, like so: Section~\ref{sec:a_section}.

\subsection{Source code}

This is a subsection.

You can typeset code samples and other verbatim text using four spaces of indentation:

\begin{verbatim}
def hello
  puts "hello, world"
end
\end{verbatim}

Softcover also comes with full support for syntax-highlighted source code using kramdown's default syntax, which combines the language name with indentation:

%= lang:ruby
\begin{code}
def hello
  puts "hello, world"
end
\end{code}

Softcover's Markdown mode also extends kramdown to support ``code fencing'' from GitHub-flavored Markdown:

%= lang:ruby
\begin{code}
def hello
  puts "hello, world!"
end
\end{code}

The last of these can be combined with \PolyTeX's \kode{codelisting} environment to make code listings with linked cross-references (Listing~\ref{code:hello}).

\begin{codelisting}
\codecaption{Hello, world.}
\label{code:hello}
%= lang:ruby
\begin{code}
def hello
  puts "hello, world!"
end
\end{code}
\end{codelisting}

\subsection{Mathematics}

Softcover's Markdown mode supports mathematical typesetting using \LaTeX\ syntax, including inline math, such as \( \phi^2 - \phi - 1 = 0, \) and centered math, such as
\[ \phi = \frac{1+\sqrt{5}}{2}. \]
It also supports centered equations with linked cross-reference via embedded \PolyTeX\ (Eq.~\eqref{eq:phi}).

\begin{equation}
\label{eq:phi}
\phi = \frac{1+\sqrt{5}}{2}
\end{equation}

Softcover also supports an alternate math syntax, such as \(\phi^2 - \phi - 1 = 0\), and centered math, such as

\[\phi = \frac{1+\sqrt{5}}{2}.\]

The \LaTeX\ syntax is strongly preferred, but the alternate syntax is included for maximum compatibility with other systems.

\section{Images and tables}

This is the second section.

Softcover supports the inclusion of images, like this:

\image{images/01_michael_hartl_headshot.jpg}

Using \LaTeX\ labels, you can also include a caption (as in Figure~\ref{fig:captioned_image}) or just a figure number (as in Figure~\ref{fig:figure_number}).

\begin{figure}[h]
\begin{center}
\image{images/01_michael_hartl_headshot.jpg}
\end{center}
\caption{Some dude.\label{fig:captioned_image}}

\end{figure}

\begin{figure}[h]
\begin{center}
\image{images/01_michael_hartl_headshot.jpg}
\end{center}
\caption{\label{fig:figure_number}}

\end{figure}

\subsection{Tables}

Softcover supports raw tables via a simple table syntax:

\begin{longtable}{|l|l|l|l|}
\hline
\textbf{HTTP request} & \textbf{URL} & \textbf{Action} & \textbf{Purpose}\\
\kode{GET} & /users & \kode{index} & page to list all users\\
\kode{GET} & /users/1 & \kode{show} & page to show user with id \kode{1}\\
\kode{GET} & /users/new & \kode{new} & page to make a new user\\
\kode{POST} & /users & \kode{create} & create a new user\\
\kode{GET} & /users/1/edit & \kode{edit} & page to edit user with id \kode{1}\\
\kode{PATCH} & /users/1 & \kode{update} & update user with id \kode{1}\\
\kode{DELETE} & /users/1 & \kode{destroy} & delete user with id \kode{1}\\
\hline
\end{longtable}

See \href{http://manual.softcover.io/book/softcover_markdown#sec-embedded_tabular_and_tables}{\emph{The Softcover Book}} to learn how to make more complicated tables.

\section{Command-line interface}

Softcover comes with a command-line interface called \kode{softcover}. To get more information, just run \kode{softcover help}:

%= lang:console
\begin{code}
$ softcover help
Commands:
  softcover build, build:all       # Build all formats
  softcover build:epub         # Build EPUB
  softcover build:html         # Build HTML
  softcover build:mobi         # Build MOBI
  softcover build:pdf          # Build PDF
  softcover build:preview          # Build book preview in all formats
  .
  .
  .
\end{code}

\noindent You can run \kode{softcover help \textless{}command\textgreater{}} to get additional help on a given command:

%= lang:console
\begin{code}
$ softcover help build
Usage:
  softcover build, build:all

Options:
  -q, [--quiet]   # Quiet output
  -s, [--silent]  # Silent output

Build all formats
\end{code}

\section{Miscellanea}

This is the end of the template---apart from two mostly empty chapters. In fact, let’s include the last chapter in its entirety, just to see how mostly empty it is:
%= <<(chapters/yet_another_chapter.md, lang: text)

Visit \href{http://manual.softcover.io}{\emph{The Softcover Book}} to learn more about what Softcover can do.